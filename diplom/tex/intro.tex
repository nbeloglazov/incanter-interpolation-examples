
\section{Введение}

С развитием компьютеров и языков программирования они всё чаще используются для научных и статистических расчётов. Для этих целей были созданы специальные системы компьютерной математики (СКМ) такие как Mathemacia, Matlab, Maple, R, Sage. Incanter - одна из таких систем написанных на языке Clojure.

Во время различных вычислений иногда возникает необходимость в интерполяции: построении непрерывных функций по заданным значениям в определённых узлах. Во всех крупных СКМ присутстсвуют специальные модули для интерполяции функций. Задачей данной работы является добавление такого модуля для интерполяции функций 1 и 2 переменных в систему Incanter.

В отчёте кратко описан язык Clojure, система Incanter, рассмотрены 4 метода интерполяции функций 1 и 2 переменных.

%%% Local Variables:
%%% mode: latex
%%% TeX-master: "../document"
%%% End:
