\section{Введение}

В последнее время бурное развитие получило новое, актуальное научное направление – компьютерная математика. Ее можно определить как совокупность теоретических, алгоритмических, аппаратных и программных средств, предназначенных для эффективного решения на компьютерной технике всех видов математических задач, включая символьные преобразования и вычисления с высокой степенью визуализации всех видов вычислений. Применение компьютерной математики существенно расширяет возможности автоматизации всех этапов математического моделирования. Существует множество различных систем компьютерной алгебры, реализующие данные возможности. К ним можно отнести Matlab, Mathematica, Maple, R, Sage, Incanter и другие.

Вышеперечисленные системы включают в себя различные подсистемы для работы с линейной алгеброй, графикой, статистикой, символьными вычислениями и другими областями вычислительной математики. Большинство таких систем предоставляют функции, позволяющие интерполировать наборы данных. Многим из тех, кто сталкивается с научными и инженерными расчётами часто приходится оперировать наборами значений, полученных опытным путём или методом случайной выборки. Как правило, на основании этих наборов требуется построить функцию, на которую могли бы с высокой точностью попадать другие получаемые значения. Такая задача называется аппроксимацией. Интерполяцией называют такую разновидность аппроксимации, при которой кривая построенной функции проходит точно через имеющиеся точки данных.

Incanter - система компьютерной алгербы, написанная на языке Clojure. В данной системе отсутствуют инструменты для интерполяции функций. Целью данной работы является изучение различных алгоритмов интерполяции функций одной и двух переменных, их реализация и добавление в систему Incanter.

%%% Local Variables:
%%% mode: latex
%%% TeX-master: "../document"
%%% End:
