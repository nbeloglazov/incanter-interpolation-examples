\section{Интерфейс библиотеки}

Разработанная библиотека предоставляет 3 функции:

\begin{minted}[gobble=2]{clojure}
  (interpolate points type & options)

  (interpolate-parametric points type & options)

  (interpolate-grid grid type & options)
\end{minted}

Все вышеперечисленные функции имеют сходную сигнатуру. Они принимают следующие параметры:

\begin{itemize}
\item \texttt{points} или \texttt{grid} - набор точек или сетка, которые требуется интерполировать;
\item \texttt{type} - тип интерполяции;
\item \texttt{options} - дополнительные опции, специфичные для каждой функции и типа интерполяции;
\end{itemize}

Все функции возвращают новую функцию - интерполирующую функцию, с помощью которой находятся значения в интересующих точках.

\subsection{Функция \texttt{interpolate}}

Данная функция позволяет построить интерполирующую функцию $f(x) = y$ по заданному набору точек $(x_i, y_i)$. Точки могут задаваться в любом порядке, перед использованием они будут автоматически отсортированы по координате $x$.

Данная функция поддерживает следующие типы интерполяции: линейная, полиномиальная, кубический сплайн, кубический Эрмитов сплайн, среднеквадратичное приближениe. Сооветствующие аргументы для параметра \texttt{type}: \texttt{:linear}, \texttt{:polynomial}, \texttt{:cubic}, \texttt{:cubic-hermite}, \texttt{:linear-least-squares}.

\subsubsection{Дополнительные опции:}
\begin{itemize}
\item \texttt{:boundaries} - граничные условия для кубического сплайна. Поддерживаются 2 вида условий: естественные (\texttt{:natural})
                             и замкнутые \\ (\texttt{:closed});
\item \texttt{:derivatives} - производные для кубического эрмитова сплайна;
\item \texttt{:basis}, \texttt{:n}, \texttt{:degree} - опции настройки среднеквадратичного приближения. Позволяют задать базис, произвольный или один из 2 встроенных (полиномиальный и B-сплайны); число функций в базисе, если выбран встроенный; степень B-сплайнов;
\end{itemize}

\subsubsection{Пример:}

Построение кубического сплайна с замкнутыми граничными условиями по точкам $(0, 0), (1, 3), (2, 0), (4, 2)$:

\begin{minted}[gobble=2,frame=single]{clojure}
  (def points [[0 0] [1 3] [2 0] [4 2]])

  (def cubic (interpolate points :cubic
                          :boundaries :closed))

  (cubic 0) ; 0.0
  (cubic 1) ; 3.0
  (cubic 3) ; -1.2380952380952381
\end{minted}

\subsection{Функция \texttt{interpolate-parametric}}

Данная функция строит интерполирующую параметрическую функцию $f(t) = (x^1, x^2, \ldots, x^n)$ определённую на отрезке $[0, 1]$ и проходяющую через заданные пользователем точки $(x_i^1, x_i^2, \ldots, x_i^n)$. Особенность данной функции заключается в том, что пользователь не задаёт узлы $t_i$, соответствующие каждой точке, данные узлы выбираются алгоритмом. Эта особенность используется в полиномиальной интерполяции, для которой очень важен выбор хороших узлов: для параметрической полиномиальной интерполяции используются узлы Чебышева.

Данная функция поддерживает такие же типы интерполяции, как \\ \texttt{interpolate}, а также B-сплайн (\texttt{:b-spline}).

\subsubsection{Дополнительные опции:}

Поддерживается тот же набор дополнительных опций, которые поддерживает функция \texttt{interpolate}, а также добавлены 2 новые опции:
\begin{itemize}
\item \texttt{:degree} - задаёт степень B-сплайна, если выбрана аппроксимация B-сплайнами;
\item \texttt{:range} - отрезок, на котором будет определена интерполирующая функция, по умолчанию это $[0, 1]$;
\end{itemize}

\subsubsection{Пример:}

Построение параметрической интерполирующей функции на отрезке $[-1, 1]$ по точкам $(0, 0), (1, 3), (2, 0), (4, 2)$:

\begin{minted}[gobble=2,frame=single]{clojure}
  (def points [[0 0] [1 3] [2 0] [4 2]])

  (def linear (interpolate-parametric points :linear
  :range [-1 1]))

  (linear -1) ; (0.0 0.0)
  (linear  1) ; (4.0 2.0)
  (linear  0) ; (1.5 1.5)
\end{minted}


\subsection{Функция \texttt{interpolate-grid}}

Данная функция строит интерполирующую функцию 2 переменных $f(x, y) = z$ по прямоугольной сетке точек. По умолчанию пользователь задаёт сетку значений $\{z_{ij}\}, i=\overline{1, n}, j=\overline{1, m}$, далее автоматически строится равномерная сетка узлов размера $n \times m$ заданная на области $[0, 1] \times [0, 1]$. В конце полученная сетка узлов и значений интерполируeтся и строится интерполирующая функция.

Функция поддерживает следующие типы интерполяции: билинейная \\ (\texttt{:bilinear}), полиномиальная (\texttt{:polynomial}), бикубический сплайн \\ (\texttt{:bicubic}), бикубический эрмитов сплайн (\texttt{:bicubic-hermite}), среднеквадратичное приближение (\texttt{:linear-least-squares}), B-сплайновая поверхность (\texttt{:b-surface}).

\subsubsection{Дополнительные опции:}

\begin{itemize}
\item \texttt{:boundaries} - граничные условия для бикубического сплайна. Поддерживаются 2 вида условий: естественные (\texttt{:natural})
                             и замкнутые (\texttt{:closed});
\item \texttt{:basis}, \texttt{:n} - опции настройки среднеквадратичного приближения. Позволяют задать базис, произвольный или встроенный (полиномиальный) и число функций в базисе, если выбран встроенный;
\item \texttt{:degree} - задаёт степень B-сплайна, если выбрана аппроксимация B-сплайновой поверхностью;
\item \texttt{:x-range}, \texttt{:y-range} - область, на которой задана интерполирующая функция, по умолчанию это $[0, 1] \times [0, 1]$;
\item \texttt{:xs}, \texttt{:ys} - явное задание узлов интерполяции;
\end{itemize}

\subsubsection{Пример:}

Интерполирование сетки значений $\begin{pmatrix} 0 & 1 & 2 \\ 3 & 4 & 5 \\ 6 & 7 & 8 \end{pmatrix}$ c использованием билинейной интерполяции:

\begin{minted}[gobble=2,frame=single]{clojure}
  (def grid [[0 1 2]
             [3 4 5]
             [6 7 8]])

  (def bilinear (interpolate-grid grid :bilinear))

  (bilinear    0   0) ; 0.0
  (bilinear    1   1) ; 8.0
  (bilinear  0.5 0.5) ; 4.0
  (bilinear 0.25   1) ; 6.5
\end{minted}




%%% Local Variables:
%%% mode: latex
%%% TeX-master: "../document"
%%% End:
