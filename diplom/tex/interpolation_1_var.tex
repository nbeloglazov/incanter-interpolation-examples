\section{Интерполяция функции одной переменной}

Рассмотрим набор попарно различных точек $\{x_i\}_{i=0}^n, x_i \in [a, b]$. Пусть $\{y_i\}_{i=0}^n$- значения некоторой функции $f \colon [a,b] \to \mathbb{R}$: в этих точках: $y_i = f(x_i)$. Предполагается, что сама функция $f$ не известна, а известны только её значения в точках $x_i$. Задача интерполяции функции 1 переменной - построить функцию $\varphi \colon [a, b] \to \mathbb{R}$, такую что выполняются следующие условия: $\varphi(x_i) = y_i$. Т.е. построенная функция $\varphi$ должна совпадать с неизвестной функцией $f$ в заданном наборе узлов. Далее будут рассмотрены 3 способа построения функции $\varphi$: линейный, полиномиальный и кубическая интерполяция. Также будут рассмотрены B-сплайны, которые не интерполируют функцию, а приближают её.

\subsection{Линейная интерполяция}

Линейная интерполяция - наиболее простой способ интерполяция, при котором $\varphi$ является кусочно-линейной функцией. При таком способе интерполяции соседние узлы соединены прямой линией. Интерполирующая функция $\varphi$ имеет следующий вид:
\begin{equation}
  \varphi(x) = y_i + (y_{i+1} - y_i) \frac{x-x_i}{x_{i+1} - x_i}, x\in [x_i, x_{i+1}]
\end{equation}

Данный метод является наиболее быстрым методом из всех представленных. Но полученная функция не является непрерывно-дифференциируемой.

\subsubsection{Временная сложность метода:}

Построение $\varphi$: $ O(n \log n)$ - требуется отсортировать все узлы.

Вычисление $\varphi(x)$: $O(\log n)$ - поиск соответствующего узла.

\subsubsection{Пример:}

\begin{minted}[gobble=2]{clojure}
  ; define points
  (def points [[0 0] [1 3] [2 0] [5 2] [6 1] [8 2] [11 1]])

  ; build interpolation function
  (def lin (interpolate points :linear))

  ; view plot on [0, 11]
  (view (function-plot lin 0 11))
\end{minted}

\imagefit{linear_interpolation_1_var}


\subsection{Полиномиальная интерполяция в форме Ньютона}

В данном способе интерполяции $\varphi$ является многочленом степени $n$. Для построение $\varphi$ используется формула Ньютона с разделёнными разностями. Вычисления производятся по следующим формулам:

\begin{equation}
  \begin{gathered}
    \varphi(x) = \sum_{i=0}^n f[x_0, \dotsc, x_i] \omega_i(x) \\
    \omega_i(x) = (x - x_0)(x - x_1) \dotsc (x - x_i) \\
    f[x_0, \dotsc, x_i] = \frac{f[x_1, \dotsc, x_i] - f[x_0, \dotsc, x_{i-1}]}{x_i - x_0}, \forall i \geq 1 \\
    f[x_j] = f(x_j), \forall j
  \end{gathered}
\end{equation}

Результатом алгоритма является многочлен, функция дифференциируемая бесконечная число раз. Недостатком данного метода является то, что многочлен может иметь большую погрешность интерполирования, которая очень сильно зависит от выбора узлов $\{x_i\}$. Особенно это проявляется при больших $n$. Другим недостаком является низкая скорость построения многочлена и вычисления его значения в точке.

\subsubsection{Временная сложность метода:}

Построение $\varphi$: $O(n^2)$

Вычисление $\varphi(x)$: $O(n)$

\subsubsection{Пример:}

\begin{minted}[gobble=2]{clojure}
  ; define points
  (def points [[0 0] [1 3] [2 0] [5 2] [6 1] [8 2] [11 1]])

  ; build interpolation function
  (def polynom (interpolate points :polynomial))

  ; view plot on [0, 11]
  (view (function-plot polynom 0 11))
\end{minted}

\imagefit{polynomial_interpolation_1_var}



\subsection{Полиномиальная интерполяция в барицентрической форме}

Другой способ построения полиномиальных интерполяционных многочленов состоит в использовании барицентрической формы записи многочлена Лагранжа. Барицентрическая интерполяционная формула выглядит следующим образом:

\begin{equation}
  \varphi(x) = \frac
  {\sum\limits_{i=0}^n y_i \frac{v_i}{x - x_i}}
  {\sum\limits_{i=0}^n \frac{v_i}{x - x_i}}
\end{equation}

\noindent где

\begin{equation}
  v_i = \frac{1}{\prod_{j \neq i}(x_i - x_j)}, \;\; i = \overline{0,n}
\end{equation}

Одна из особенностей барицентрической формы состоит в том, что для построение многочлена для нового набора ${y_i}$ не требует пересчёта коэффициентов, как в форме Ньютона. Данная особенность хорошо подходит для построение параметрических кривых, когда узлы параметра $t$ фиксируются, вычисляются $v_i$, а далее поочередно подставляются наборы ${x_i}, {y_i}, {z_i}$ и получаются многочлены по каждому из наборов.

Если использовать узлы специального вида, то вычисление $v_i$ может свестись к более простой форме. Например если $x_i$ - чебышевские узлы второго рода:

\begin{equation}
  x_i = \cos \frac{i \pi}{n}, \;\; i = \overline{0, n}
\end{equation}

\noindent тo коэфициенты $v_i$ имеют следующий вид:

\begin{equation}
  v_i = (-1)^i \delta_i, \;\;
  \delta_i = \left\{
      \begin{array}{l l}
        1/2, & \text{ если } i = 0 \text{ или } i = n, \\
        1, & \text{ иначе}.
      \end{array}
      \right.
\end{equation}

Барицентрическая форма использовалась для параметрической интерполяции, когда пользователь не задаёт узлы, а задаёт только точки, через которые должна проходить кривая и интервал параметризации. В качестве узлов использовались чебышевские узлы второго рода.

\subsubsection{Временная сложность метода:}

Построение $\varphi$: $O(n)$ - при условии использовании специальных узлов и быстрого вычисления $v_i$ за $O(1)$.

Вычисление $\varphi(x)$: $O(n)$

\subsubsection{Пример:}

\begin{minted}[gobble=2]{clojure}
  ; define points
  (def points [[0 0] [1 3] [2 0] [5 2] [6 1] [8 2] [11 1]])

  ; build interpolation function
  (def barycentric (interpolate-parametric points :polynomial))

  ; view plot on [0, 1]
  (view (parametric-plot barycentric 0 1))
\end{minted}

\imagefit{polynomial_barycentric_interpolation_1_var}



\subsection{Кубический сплайн}

Функция $\varphi$ является кусочной и на каждом отрезке $[x_{i-1},x_i], i=\overline{1,n}$ задаётся отдельным кубическим многочленом:
\begin{equation}
  \varphi(x) = s_i(x) = \alpha_i + \beta_i(x - x_i) + \frac{\delta_i}{2}(x - x_i)^2 + \frac{\delta_i}{6}(x - x_i)^3, x \in [x_{i-1}, x_i]
\end{equation}

Также накладываются требования наличия непрерывной первой и второй производной $\varphi$, из чего получаем дополнительные условия:

\begin{equation}
  \begin{gathered}
    s'_{i-1}(x_{i-1}) = s'_i(x_{i-1}), i = \overline{2, n} \\
    s''_{i-1}(x_{i-1}) = s''_i(x_{i-1}), i = \overline{2, n}
  \end{gathered}
\end{equation}

Используя эти условия и условия интерполяции получаем следующие формулы для вычисления коэффициентов $\{\alpha_i\}$, $\{\beta_i\}$, $\{\delta_i\}$:

\begin{equation}
  \begin{gathered}
    h_i = x_i - x_{i-1}, i = \overline{1, n} \\
    \alpha_i = y_i, i = \overline{0, n} \\
    \beta_i = \frac{y_i - y_{i-1}}{h_i} + \frac{2\gamma_i + \gamma_{i - 1}}{6}, i = \overline{1, n} \\
    \delta_i = \frac{\gamma_i - \gamma_{i-1}}{h_i}, i = \overline{2, n}
  \end{gathered}
\end{equation}

Коэффициенты $\{\delta_i\}$ можно получить, решив следующую 3-диагональную систему линейных уравненией:

\begin{equation}
  h_i \gamma_{i-1} + 2(h_i + h_{i+1})\gamma_i + h_{i+1}\gamma_{i+1} =
  6 (\frac{y_{i+1} - y_i}{h_{i-1}} - \frac{y_i - y_{i-1}}{h_i}), i = \overline{i, n - 1}
\end{equation}

Данная система имеет $n-2$ уравнений и $n$ неизвестных. Задавая различные граничные условия можно получить недостающие уравнения и решить систему, тем самым получая коэффициенты для кубического сплайна. Были реализованы 2 вида граничных условий:

\begin{enumerate}
\item Естественные граничные условия. Полагают $\varphi''(x_0)= \varphi''(x_n)=0$.
\item Периодические (замкнутые) граничные условия. Полагают $\varphi'(x_0)=\varphi'(x_n),\, \varphi''(x_0)=\varphi''(x_n)$.
\end{enumerate}

\subsubsection{Временная сложность метода:}

Построение $\varphi$: $O(n)$

Вычисление $\varphi(x)$: $O(\log n)$ - поиск соответствующего промежутка.

\subsubsection{Пример:}

\begin{minted}[gobble=2]{clojure}
  ; define points
  (def points [[0 0] [1 3] [2 0] [5 2] [6 1] [8 2] [11 1]])

  ; build interpolation function
  (def cubic (interpolate points :cubic :boundaries :closed))

  ; view plot on [0, 11]
  (view (function-plot cubic 0 11))
\end{minted}

\imagefit{cubic_interpolation_1_var}


\subsection{Кубический эрмитов сплайн}

Кусочно-полиномиальная функция $\varphi$ называется эрмитовым сплайном третьей степени для $f \in C^1[a, b]$, если
\begin{equation}
  \begin{gathered}
    s\mid_{x\in \Delta_i} = s_i \in \mathbb{P}_3 \\
      s(x_i) = f(x_i) \text{ и } s'(x_i) = f'(x_i) \;\; \forall i = \overline{0,n}
  \end{gathered}
\end{equation}

Каждая функция $s_i$ является интерполяционным многочленом Эрмита третьей степени. Вычислять его будем используя форму Ньютона:

\begin{equation}
  s_i(x) = \sum_{j=0}^3 \alpha_{ij} \omega_{ij}(x)
\end{equation}

\noindent Где

\begin{equation}
  \begin{matrix*}[l]
    \alpha_{i0} = f(x_{i-1}) & \omega_{i0}(x) = 1\\
    \alpha_{i1} = f'(x_{i-1}) & \omega_{i1}(x) = x - x_{i-1} \\
    \alpha_{i2} = f[x_{i-1}, x_{i-1}, x_i] & \omega_{i2}(x) = (x - x_{i-1})^2 \\
    \alpha_{i3} = f[x_{i-1}, x_{i-1}, x_i, x_i] & \omega_{i3}(x) =  (x - x_{i-1})^2(x-x_i)
  \end{matrix*}
\end{equation}

\noindent Первыe производные функции в узлах $x_i$ приближались используя конечные разности:

\begin{equation}
  f'(x_i) \approx \left\{
  \begin{array}{l l}
    \frac{f(x_1) - f(x_0)}{x_1 - x_0} & i = 0 \\
    \frac{1}{2}(\frac{f(x_i) - f(x_{i-1})}{x_i - x_{i-1}} + \frac{f(x_{i+1}) - f(x_i)}{x_{i+1} - x_i}) & i = \overline{1, n-1} \\
    \frac{f(x_n) - f(x_{n-1})}{x_n - x_{n-1}} & i = n
  \end{array}
  \right.
\end{equation}

\subsubsection{Временная сложность метода:}

Построение $\varphi$: $O(n)$

Вычисление $\varphi(x)$: $O(\log n)$ - поиск соответствующего промежутка.

\subsubsection{Пример:}

\begin{minted}[gobble=2]{clojure}
  ; define points
  (def points [[0 0] [1 3] [2 0] [5 2] [6 1] [8 2] [11 1]])

  ; build interpolation function
  (def hermite (interpolate points :cubic-hermite))

  ; view plot on [0, 11]
  (view (function-plot hermite 0 11))
\end{minted}

\imagefit{cubic_hermite_interpolation_1_var}


\subsection{B-сплайн}

Пусть задана бесконечная сетка узлов ${x_i}$:

\[
\ldots < x_{-1} < x_0 < x_1 < \ldots
\]

Тогда B-сплайном порядка $m \geq 1$ называется функция $N^m_i$, определяемая соотношением:

\begin{equation}
  \begin{gathered}
    N^m_i = V^m_i N^{m-1}_i + (1 - V^m_{i+1}) N^{m-1}_{i+1}, \forall i \in \mathbb{Z} \\
    V^m_i(x) = \frac{x - x_i}{x_{i+m} - x_i}
  \end{gathered}
\end{equation}

B-сплайны имеют небольшой носитель: $supp N^m_i = [x_i, x_{i+m+1}]$. Таким образом, если мы будем строить функцию, являющуюся линейной комбинацией B-сплайнов, то изменения коэффициента при одном из сплайнов повлияет лишь не небольшую часть функцию, изменение будет локальным.

При переходе от бесконечной сетки $\{x_i\}$ к конечной сетке, заданной на отрезке $[a, b]$ мы сталкиваемся с проблемой, что носитель некоторых базисных функций не входит полностью в $[a, b]$. Чтобы решить эту проблему, добавим виртуальные узлы, равные $a$ слева и $b$ справа. В результате в изменённой сетке появились кратные узлы, чтобы избежать деления на 0 добавим условие, что значение $V^m_i(x)$ будет равно $0$, если знаменатель равен 0.

Рассмотрим приложение B-сплайнов к задаче интерактивного дизайна кривой.

Пусть $\{q_i\}^M_{i=0}$ - контрольные точки на плоскости. Построим обобщение кривой Безье для этих контрольных точек, используя в качестве базиса B-сплайны. В качестве отрезка параметризации возьмём $[a, b] = [0, 1]$. Требуется $M + 1$ линейно независимых на этом отрезке базисных спалйнов порядка $m$. Для этого возьмём сетку равномерно расположенных узлов с добавленными виртуальными узлами:

\begin{equation}
  X = \{\underbrace{0, \ldots, 0}_{m+1}, \frac{1}{n},\frac{2}{n}, \ldots, \frac{n-1}{n}, \underbrace{1, \ldots, 1}_{m+1}\}
\end{equation}
Зададим функцию $\varphi(x)$ в виде линейной комбинации B-сплайнов, построенных по заданным узлам:

\begin{equation}
  \varphi(t) = \sum_{i=0}^M q_i N^m_i(t)
\end{equation}

\subsubsection{Временная сложность метода:}

Построение $\varphi$: $O(n)$

Вычисление $\varphi(t)$: $O(m^2)$, где $m$ - степень сплайна


\subsubsection{Пример:}

\begin{minted}[gobble=2]{clojure}
  ; define points
  (def points [[0 0] [1 3] [2 0] [5 2] [6 1] [8 2] [11 1]])

  ; build approximation function
  (def b-spline (interpolate-parametric points :b-spline :degree 3))

  ; view plot on [0, 1]
  (view (parametric-plot cubic 0 1))
\end{minted}

\imagefit{b_spline_approximation_1_var}


\subsection{Среднеквадратичное приближение}

Среднеквадратичное приближение, как и B-сплайны, апроксимирует заданные узлы, а не интерполирует их. Задача дискретного среднеквадратичного приближения для данного набора точек $(x_k, y_k), k = \overline{0, N}$, заключается в построении функции $\varphi$ вида:

\begin{equation}
  \varphi(x) = \sum^n_{i=0} \alpha_i \varphi_i(x)
\end{equation}

\noindent для которой выражение

\begin{equation}
  \sum^N_{k=0}(y_k - \varphi(x_k))^2
\end{equation}

\noindent принимает минимальное возможное значение. Здесь $\{\varphi_i(x)\}^n_{i=0}$ - некоторый заданный базис.

Раскрыв данную формулу получаем квадратичную форму. После минимизации полученной квадратичной формы мы приходим к тому, что коэффициенты $\{\alpha_i\}$ можно найти решив систему линейных уравнений:

\begin{equation}
  \sum^n_{j=0} \gamma_{lj} \alpha_j = \beta_j, \;\;\; l = \overline{0, n}
\end{equation}

\noindent где:

\begin{equation}
  \begin{gathered}
    \gamma_{ij} = \sum^N_{k=0} \varphi_i(x_k) \varphi_j(x_k) \\
    \beta_i = \sum^N_{k=0} y_k \varphi_i(x_k)
  \end{gathered}
\end{equation}

Решая данную систему мы получаем коэффициенты $\{\alpha_i\}$ и таким образом получаем функцию $\varphi$.

\subsubsection{Временная сложность метода:}

Построение $\varphi$: $ O(n^2N + n^3)$ - построение матрицы коэффициентов $\gamma_{ij}$ и решение СЛАУ.

Вычисление $\varphi(x)$: $O(n)$.

\subsubsection{Пример:}

\begin{minted}[gobble=2]{clojure}
  ; define points
  (def points [[0 0] [1 3] [2 0] [5 2] [6 1] [8 2] [11 1]])

  ; build approximation function
  (def lls (interpolate points :linear-least-squares
                               :basis :polynomial :n 3))

  ; view plot on [0, 11]
  (view (function-plot lls 0 11))
\end{minted}

\imagefit{lls_polynomial_3_1_var}



%%% Local Variables:
%%% mode: latex
%%% TeX-master: "../document"
%%% End:
