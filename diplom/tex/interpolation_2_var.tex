\section{Интерполяция функции двух переменных}

Задача приближения функции одной переменной, которая рассматривалась до сих пор, естественным образом обобщается на случай функций нескольких переменных. Рассмотрим задачу интерполяции функций двух переменных на прямоугольной области. Дано: набор узлов $(x_i,y_j), i=\overline{0,n}, j=\overline{0,m}$ и значения неизвестной функции $f$ в этих узлах: $f(x_i,y_j)=z_{ij}$. Требуется построить функцию удовлетворяющую условиям интерполяции: $(x_i,y_j) = z_{ij}$
Существует и более сложная версия интерполяции функции 2 переменных, в которой узлы задаются не в виде сетки, а  произвольным образом. Но решение такой задачи в данной работе рассматриваться не будет.

\subsection{Билинейная интерполяция}

Ключевая идея заключается в том, чтобы провести обычную линейную интерполяцию сначала по одной переменной, затем по другой. \\
\noindent Пусть $x \in [x_i,x_{i+1}], y \in [y_j,y_{j+1}]$. Тогда
\begin{equation}
  \begin{gathered}
    \varphi(x, y_j) = z_{i,j} + (z_{i+1,j} - z_{i,j}) \frac{x-x_i}{x_{i+1} - x_i} \\
    \varphi(x, y_{j+1}) = z_{i,j+1} + (z_{i+1,j+1} - z_{i,j+1}) \frac{x-x_i}{x_{i+1} - x_i} \\
    \varphi(x,y) = \varphi(x,y_j) + (\varphi(x, y_{j+1}) - \varphi(x, y_j)) \frac{y-y_j}{y_{j+1} - y_j}
  \end{gathered}
\end{equation}

\subsubsection{Временная сложность метода:}

Построение $\varphi$: $O(n m)$ - требуется преобразовать входную сетку для возможности быстрого поиска нужного сегмента

Вычисление $\varphi(x, y)$: $O(\log n + \log m)$ - поиск сегмента

\subsubsection{Пример:}

\imagefit{bilinear_interpolation_2_var}


\subsection{Полиномиальная интерполяция}

Полиномиальная интерполяция функции 2 переменных, как и для 1-мерного случая, была реализована в форме Ньютона.

\subsubsection{Временная сложность метода:}

Построение $\varphi$: $O(n m \max(n, m))$ - вычисление разделённых разностей для двумерного случая.

Вычисление $\varphi(x, y)$: $O(n m)$

\subsubsection{Пример:}

\imagefit{polynomial_interpolation_2_var}


\subsection{Бикубический сплайн}

Аналог кубического сплайна. Основная идея заключается в том, что фиксируется одна из переменных, например $y$,и вычисляются коэффициенты кубических сплайнов по $x$. Далее по полученным коэффициентам строятся кубические сплайны, фиксируя уже переменную $x$. Полученные сплайны и используются для вычисления функции.

\subsubsection{Временная сложность метода:}

Построение $\varphi$: $O(n m)$

Вычисление $\varphi(x, y)$: $O(\log{n} + \log{m})$

\subsubsection{Пример:}

\imagefit{bicubic_interpolation_2_var}


\subsection{B-поверхность}

B-поверхность для функции двух переменных строится используя тензорное произведение B-сплайнов для одномерного случая.

\subsubsection{Временная сложность метода:}

Построение $\varphi$: $O(n m)$

Вычисление $\varphi(x, y)$: $O(d^2)$, где $d$ - степень одномерных сплайнов.

\subsubsection{Пример:}

\imagefit{b_surface_approximation_2_var}





%%% Local Variables:
%%% mode: latex
%%% TeX-master: "../document"
%%% End:
