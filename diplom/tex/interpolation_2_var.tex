\section{Интерполяция функции двух переменных}

Задача приближения функции одной переменной, которая рассматривалась до сих пор, естественным образом обобщается на случай функций нескольких переменных. Рассмотрим задачу интерполяции функций двух переменных на прямоугольной области. Дано: набор узлов $(x_i,y_j), i=\overline{0,n}, j=\overline{0,m}$ и значения неизвестной функции $f$ в этих узлах: $f(x_i,y_j)=z_{ij}$. Требуется построить функцию удовлетворяющую условиям интерполяции: $(x_i,y_j) = z_{ij}$
Существует и более сложная версия интерполяции функции 2 переменных, в которой узлы задаются не в виде сетки, а  произвольным образом. Но решение такой задачи в данной работе рассматриваться не будет. Методы реализованы на основе лекций Б.В. Фалейчика \cite{faleichik2010} \cite{faleichik2012}.

\subsection{Билинейная интерполяция}

Ключевая идея заключается в том, чтобы провести обычную линейную интерполяцию сначала по одной переменной, затем по другой. \\
\noindent Пусть $x \in [x_i,x_{i+1}], y \in [y_j,y_{j+1}]$. Тогда
\begin{equation}
  \begin{gathered}
    \varphi(x, y_j) = z_{i,j} + (z_{i+1,j} - z_{i,j}) \frac{x-x_i}{x_{i+1} - x_i} \\
    \varphi(x, y_{j+1}) = z_{i,j+1} + (z_{i+1,j+1} - z_{i,j+1}) \frac{x-x_i}{x_{i+1} - x_i} \\
    \varphi(x,y) = \varphi(x,y_j) + (\varphi(x, y_{j+1}) - \varphi(x, y_j)) \frac{y-y_j}{y_{j+1} - y_j}
  \end{gathered}
\end{equation}

\subsubsection{Временная сложность метода:}

Построение $\varphi$: $O(n m)$ - требуется преобразовать входную сетку для возможности быстрого поиска нужного сегмента

Вычисление $\varphi(x, y)$: $O(\log n + \log m)$ - поиск сегмента

\subsubsection{Пример:}

\begin{minted}[gobble=2,frame=single]{clojure}
  ; define grid
  (def grid [[0 1 2]
             [3 4 5]
             [6 7 8]])

  ; build interpolation function
  (interpolate-grid grid :bilinear)
\end{minted}

\imagefit{bilinear_interpolation_2_var}{Билинейная интерполяция}


\subsection{Полиномиальная интерполяция}

Полиномиальная интерполяция функции 2 переменных была реализована по алгоритму, предложенному Varsamis и Karampetaki \cite{varsamis2011}. Матрица разделённых разностей $P$ вычисляется по следующей формуле:

\begin{equation}
  P^{(k)}_{i,j} =
  \begin{dcases}
    f(x_i, y_j) & \text{if  } k = 0 \\
    \frac{P^{(k-1)}_{i,j} - P^{(k-1)}_{i-1,j}}{x_i - x_{i-k}} & \text{if  } (j \leq k - 1 \land i > k - 1) \\
    \frac{P^{(k-1)}_{i,j} - P^{(k-1)}_{i,j-1}}{y_j - y_{j-k}} & \text{if  } (i \leq k - 1 \land j > k - 1) \\
    \frac{P^{(k-1)}_{i,j} + P^{(k-1)}_{i-1,j-1} - P^{(k-1)}_{i,j-1} - P^{(k-1)}_{i-1,j}}{(x_i - x_{i-k})(y_j - y_{j-k})} & \text{if  } (i > k-1 \land j > k-1)
  \end{dcases}
\end{equation}

Интерполяционный многочлен вычисляется по формуле:

\begin{equation}
  \varphi(x, y) = Y^T P X
\end{equation}

\noindent где

\begin{equation}
  \begin{gathered}
    X =
    \begin{pmatrix}
      1 \\ x - x_0 \\ \vdots \\ (x - x_0)(x - x_1) \hdots (x - x_{n-1})
    \end{pmatrix}
    \\
    Y =
    \begin{pmatrix}
      1 \\ y - y_0 \\ \vdots \\ (y - y_0)(y - y_1) \hdots (y - y_{m-1})
    \end{pmatrix}
  \end{gathered}
\end{equation}

\subsubsection{Временная сложность метода:}

Построение $\varphi$: $O(n m \max(n, m))$ - вычисление разделённых разностей для двумерного случая.

Вычисление $\varphi(x, y)$: $O(n m)$

\subsubsection{Пример:}

\begin{minted}[gobble=2,frame=single]{clojure}
  ; define grid
  (def grid [[0 1 2]
             [3 4 5]
             [6 7 8]])

  ; build interpolation function
  (interpolate-grid grid :polynomial)
\end{minted}

\imagefit{polynomial_interpolation_2_var}{Полиномиальная интерполяция}


\subsection{Бикубический сплайн}

Бикубический сплайн является аналогом кубического сплайна. Будем искать функцию $\varphi(x, y)$ будем искать в виде:

\begin{equation} \label{eq:bicubic_gen}
  \varphi(x, y) = \alpha_i(y) + \beta_i(y)(x - x_i) + \frac{\delta_i(y)}{y}(x - x_i)^2 + \frac{\delta_i(y)}{6}(x - x_i)^3, x \in [x_{i-1}, x_i]
\end{equation}

Здесь коэффициенты $\alpha_i(y), \beta_i(y), \delta_i(y), \gamma_i(y), i = \overline{1, n}$ являются одномерными кубическими сплайнами. Для построения функции $\varphi$ необходимо построить данные одномерные сплайны. Но для их построения не хватает значений $\alpha_i(y), \beta_i(y), \delta_i(y), \gamma_i(y)$ в узлах $y_j, j = \overline{1,m}$.

Чтобы решить эту проблему зафиксируем $y = y_0$ в \eqref{eq:bicubic_gen} и получаем одномерный кубический сплайн, вычисляя коэффициенты которого находим $\alpha_i(y_0), \beta_i(y_0), \delta_i(y_0), \gamma_i(y_0), i = \overline{1, n}$. Таким образом фиксируя $y$ можно получить значения коэффициентов по всех остальных узлах. Далее по значениям коэффициентов в узлах строим сплайны $\alpha_i(y), \beta_i(y), \delta_i(y), \gamma_i(y)$.

Для вычисления значения $\varphi$ в точке $(x,y)$ необходимо по $x$ определить отрезок $[x_{i-1}, x_i]$, которому принадлежит $x$. Далее требуется посчитать значения сплайнов $\alpha_i, \beta_i, \delta_i, \gamma_i$ в точке $y$. Полученные коэффициенты использовать в формуле \eqref{eq:bicubic_gen} для получения конечного результата.

\subsubsection{Временная сложность метода:}

Построение $\varphi$: $O(n m)$

Вычисление $\varphi(x, y)$: $O(\log{n} + \log{m})$

\subsubsection{Пример:}

\begin{minted}[gobble=2,frame=single]{clojure}
  ; define grid
  (def grid [[0 1 2]
             [3 4 5]
             [6 7 8]])

  ; build interpolation function
  (interpolate-grid grid :bicubic)
\end{minted}

\imagefit{bicubic_interpolation_2_var}{Бикубический сплайн}

\subsection{Средневадратическое приближение}

Среднеквадратическое приближение функции 2 переменных очень сильно похоже на среднеквадратическое приближения функции 1 переменной. Задаётся базис функций 2 переменных и вычисляются коэффициенты $\{\alpha_i\}$ при которых сумма расстояний от полученной поверхности до заданных точек минимальна. Положим $N \times M$ - размеры сетки, $n$ - число базисных функций. Изменяются только формулы вычисления $\gamma_{ij}$ и $\beta_i$:

\begin{equation}
  \begin{gathered}
    \gamma_{ij} = \sum^N_{k=0}\sum^M_{l=0} \varphi_i(x_k,y_l) \varphi_j(x_k,y_l) \\
    \beta_i = \sum^N_{k=0}\sum^M_{l=0} y_{kl} \varphi_i(x_k,y_l)
  \end{gathered}
\end{equation}

\subsubsection{Временная сложность метода:}

Построение $\varphi$: $ O(n^2NM + n^3)$ - построение матрицы коэффициентов $\gamma_{ij}$ и решение СЛАУ.

Вычисление $\varphi(x)$: $O(n)$.

\subsubsection{Пример:}

\begin{minted}[gobble=2,frame=single]{clojure}
  ; define grid
  (def grid [[0 1 2]
             [3 4 5]
             [6 7 8]])

  ; basis consists of 7 funtions: 1, x, y, sin x, cos x, sin y, cos y
  (defn basis [x y] [1 x y
                     (Math/sin x) (Math/cos x)
                     (Math/sin y) (Math/cos y)])

  ; build interpolation function
  (interpolate-grid grid :linear-least-squares
                         :basis basis)
\end{minted}

\imagefit{lls_1_x_y_sin_sin_cos_cos}{Среднеквадратичное приближение по базису $1, x, y, sin(x), cos(x), sin(y), cos(y)$}


\subsection{B-сплайновая поверхность}

B-сплайновая поверхность для функции двух переменных строится используя тензорное произведение B-сплайнов для одномерного случая. Формула $\varphi(u, v)$ выглядит следующим образом:

\begin{equation}
  \varphi(u, v) = \sum_{i=0}^m \sum_{j=0}^n q_{ij} N^d_i(u)N^d_j(v)
\end{equation}

\subsubsection{Временная сложность метода:}

Построение $\varphi$: $O(n m)$

Вычисление $\varphi(u, v)$: $O(d^2)$, где $d$ - степень одномерных сплайнов.

\subsubsection{Пример:}

\begin{minted}[gobble=2,frame=single]{clojure}
  ; define grid
  (def grid [[0 1 2]
             [3 4 5]
             [6 7 8]])

  ; build interpolation function
  (interpolate-grid grid :b-surface)
\end{minted}

\imagefit{b_surface_approximation_2_var}{B-сплайновая поверхность}





%%% Local Variables:
%%% mode: latex
%%% TeX-master: "../document"
%%% End:
