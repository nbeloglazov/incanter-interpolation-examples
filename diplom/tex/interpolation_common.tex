\section{Интерполяция}

Интерполяция - способ нахождения промежуточных значений величины по имеющемуся дискретному набору известных значений. В научных и инженерных расчётах часто приходится оперировать наборами значений, полученных опытным путём или методом случайной выборки. Как правило, на основании этих наборов требуется построить функцию, на которую могли бы с высокой точностью попадать другие получаемые значения. Такая задача называется аппроксимацией. Интерполяцией называют такую разновидность аппроксимации, при которой кривая построенной функции проходит точно через имеющиеся точки данных.

Интерполяция может применяться для различных целей, например:

\begin{itemize}
\item масштабирование изображений;

\item генерация гладких кривых и поверхностей в 3D графике;

\item цифровая обработка сигналов;

\item приближённое решение уравнений;
\end{itemize}

Задачей данной работы является создание модуля interpolation для платформы Incanter, который будет содержать функции для интерполяции функций 1 и 2 переменных.

%%% Local Variables:
%%% mode: latex
%%% TeX-master: "../document"
%%% End:
