\documentclass[russian]{beamer}

\usepackage{cmap} % для кодировки шрифтов в pdf
\usepackage[T2A]{fontenc}
\usepackage[utf8]{inputenc}
\usepackage[russian]{babel}
\usepackage{minted} % code highlighting

\usepackage{graphicx} % для вставки картинок
\graphicspath{ {img/} }

%\usetheme{Warsaw}
%\usecolortheme{beaver}

\title[Интерполяция]{Алгоритмы интерполяции функций.\newline Создание библиотеки на языке Clojure.}
\author{Белоглазов Никита}
\date{2013}
\institute{Белорусский Государственный Университет}

\begin{document}

\maketitle
%\frame{\titlepage}
\begin{frame}
  \frametitle{This is the first slide}
  hello
  \pause
  world
  % Content goes here
\end{frame}
\begin{frame}[fragile]
  \frametitle{This is the second slide}
  \framesubtitle{A bit more information about this}
  % More content goes here
  \begin{minted}{clojure}
    ; define points
    (def points [[0 0] [1 3] [2 0] [5 2] [6 1] [8 2] [11 1]])

    ; build interpolation function
    (def lin (interpolate points :linear))

    ; view plot on [0, 11]
    (view (function-plot lin 0 11))
  \end{minted}
\end{frame}

\end{document}

%%% Local Variables:
%%% mode: latex
%%% TeX-master: t
%%% End:
